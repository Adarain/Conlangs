\documentclass[paper=6in:9in]{scrbook}
\usepackage[margin=1in]{geometry}
\usepackage{color}
\usepackage{soul}
\usepackage{tabu}
\usepackage{multirow}
\usepackage{fontspec}
\usepackage[hidelinks]{hyperref}
\usepackage{float}
\usepackage{multicol}
\restylefloat{table}
\usepackage{placeins}
\usepackage{graphicx}
\usepackage{forest}
\usepackage{lingmacros}
\usepackage{vowel}
\usepackage{calc}
\usepackage{amsmath}
\usepackage{enumitem}
\usepackage{tikz, tikz-qtree}
\usetikzlibrary{positioning,decorations.markings}
\graphicspath{ {images/} }
\renewcommand*\sectfont{\normalcolor\bfseries}

\newfontfamily\lib{Linux Libertine}
\setmainfont[Ligatures=TeX]{Charis SIL}
\definecolor{grey}{rgb}{0.84, 0.84, 0.84}
\newcommand{\gl}[1]{\textsc{#1}}
\newcommand{\en}[1]{``#1''}
\newcommand{\con}[1]{\hspace{0pt}{\color{olive}#1}}
\renewcommand{\ex}[1]{\con{#1}\\}
\newcommand{\orth}[1]{{\lib{}⟨}#1{\lib{}⟩}}
\newcommand{\broad}[1]{/#1/}
\newcommand{\narrow}[1]{[#1]}
\newcommand{\langname}{Semụr}
\newcommand{\culturename}{Semuru}

\newcommand{\circled}[1]{%
    \tikz[baseline=(char.base)]{\node[shape=circle,draw,inner sep=0.1ex] (char) {#1};}%
}

\newcommand{\qdot}[0]{q̇}
\renewcommand{\cdot}[0]{ċ}

\title{\langname}
\author{Sascha M. Baer}
\date{}

\begin{document}

\maketitle
\newpage

\tableofcontents
\listoftables

\newpage
\chapter{Phonology}
The \langname{} language has 21 consonant phonemes and 5 vowel phonemes, shown in tables \ref{consint} and \ref{vowint}. Several important allophones can be identified too, however, which are listed in those tables in parentheses. Their occurences will be described in section \ref{sec:allophony}. 

\begin{table}[]
\centering
\begin{tabular}{l|ccc|ccc|ccc}
                   & \multicolumn{3}{c|}{Labial}      & \multicolumn{3}{c|}{Alveolar}    & \multicolumn{3}{c}{Velar}        \\ \hline
\multirow{2}{*}{N}       &           & m̥        & m        &           & n̥        & n        &           &   (ŋ̊)     &    (ŋ)   \\
                             &           & \orth{hm} & \orth{m} &           & \orth{hn} & \orth{n} &           & \orth{hn} & \orth{n, \qdot}  \\
\multirow{2}{*}{S}        & pʰ        & p         & b        & tʰ        & t         & d        & kʰ        & k         & g        \\
                             & \orth{ph} & \orth{p}  & \orth{b} & \orth{th} & \orth{t}  & \orth{d} & \orth{kh} & \orth{k}  & \orth{g} \\
\multirow{2}{*}{Af}   &           &           &          & tsʰ       & ts       &          & kxʰ       & kx       &          \\
                             &           &           &          & \orth{ch} & \orth{c}  &          & \orth{qh} & \orth{q}  &          \\
\multirow{2}{*}{F}   &           & f         &          &           & s         &          &           &           &          \\
                             &           & \orth{f}  &          &           & \orth{s}  &          &           &           &          \\
\multirow{2}{*}{T}         &           &           &          &           &           & ɾ        &           &           &          \\
                             &           &           &          &           &           & \orth{r, \cdot} &           &           &          \\
\multirow{2}{*}{Ap} &           &           & ʋ        &           &           & l        &           &           &          \\
                             &           &           & \orth{v} &           &           & \orth{l} &           &           &         
\end{tabular}
\caption{Consonant Inventory}
\label{consint}
\end{table}

\begin{table}[]
\centering
\large
\begin{vowel}
\putcvowel{i}{1}
\putcvowel{ɛ}{3}
\putcvowel{a}{4}
\putcvowel[r]{ɔ}{6}
\putcvowel[r]{u}{8}
\putcvowel[l]{(ʌ)}{6}
\putcvowel[l]{(ɯ)}{8}
\putcvowel{(ə̥)}{11}
\end{vowel}
\begin{vowel}
\putcvowel{\orth{i}}{1}
\putcvowel{\orth{e}}{3}
\putcvowel{\orth{a}}{4}
\putcvowel[r]{\orth{o}}{6}
\putcvowel[r]{\orth{u}}{8}
\putcvowel[l]{\orth{ọ}}{6}
\putcvowel[l]{\orth{ụ}}{8}
\putcvowel{\orth{◌̆}}{11}
\end{vowel}
\caption{Vowel Inventory}
\label{vowint}
\end{table}


\begin{table}[]
\centering

\begin{tabular}{l|ccc|ccc|ccc}
                       & \multicolumn{3}{c|}{Alveolar}       & \multicolumn{3}{c|}{Lateral}        & \multicolumn{3}{c}{Palatal}        \\ \hline
\multirow{2}{*}{Nasal} &           & ᵑ̊!        & ᵑ!         &           & ᵑ̊ǁ        & ᵑǁ         &           & ᵑ̊ǂ        & ᵑǂ         \\
                       &           & \orth{n!k} & \orth{n!g} &           & \orth{nǁk} & \orth{nǁg} &           & \orth{nǂk} & \orth{nǂg} \\
\multirow{2}{*}{Oral}  & ᵏ!ʰ       & ᵏ!         & ᵍ!         & ᵏǁʰ       & ᵏǁ         & ᵍǁʰ        & ᵏǂʰ       & ᵏǂ         & ᵍǂʰ        \\
                       & \orth{!q} & \orth{!k}  & \orth{!g}  & \orth{ǁq} & \orth{ǁk}  & \orth{ǁg}  & \orth{ǂq} & \orth{ǂk}  & \orth{ǂg} 
\end{tabular}
\caption{Click Inventory}
\label{clickint}
\end{table}
A threefold distinction is found in plosives: tenuis, voiced and aspirated. A voicing distinction also occurs in nasals, while aspiration is distinguished for affricates. All other manners of pronunciation only have one member: Non-sibilant fricative \broad{f}, sibilant fricative \broad{s}, tap \broad{ɾ}, approximant \broad{ʋ} and lateral approximant \broad{l}.

Among places of articulation it is noticable that the velar column is smaller than the other two. The velar nasals, while not uncommon, only occur as allophones of other sounds. 


\section{Syllable Structure}

A distinction may be first drawn between light (monomoraic) and heavy (bimoraic) syllables. Heavy syllables are those which have a coda consonant. Syllable weight is important for determination of stress and accent pattern (see \ref{sec:stressaccent}).

The structure of light syllables is always any consonant followed by any vowel. There are no restrictions on the combination of these phonemes. 

A heavy syllable has the same structure as a light one, except that is also followed by a coda consonant. The list of phonemes occuring in codas is restricted: only \broad{s ɾ l} as well as all nasals may occur in the coda of a syllable. 

\section{Allophony} \label{sec:allophony}


\section{Stress and Accent} \label{sec:stressaccent}
Stress placement within a word is entirely predictable from its segments. 

\chapter{Noun Phrases}

A \langname{} noun phrase typically consists of two words: an article and a noun. 





\end{document}