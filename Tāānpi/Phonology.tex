\section{Phonology}
\emph{Monday, May 14th 2018}\\

\noindent This first paper is intended to provide a thorough description of the \lang{} phonology, including sounds, word structure, tonology and also a bit of a historical outlook explaining the described phenomena diachronically.

\subsection{Phoneme inventory}
The following table lists all consonant phonemes and their romanization found in \lang{}:
\begin{table}[H]
\centering
\begin{tabular}{r|l|lll|l|l}
Phonemes           & \multicolumn{1}{c|}{Lab} & \multicolumn{3}{c|}{Alv} & \multicolumn{1}{c|}{Vel} & \multicolumn{1}{c}{Glot} \\ \hline
N                  & m                        & n      &        &        &                          &                          \\
\multirow{2}{*}{P} & pʰ                       & tʰ     & tsʰ    & tɕʰ    &                          &                          \\
                   & p                        & t      & ts     & tɕ     & k                        & ʔ                        \\
F                  & ɸ                        &        & s      & ɕ      &                          & h                        \\
L                  &                          &        & l      & j      & ɰ                        & \\
\multicolumn{1}{c}{}\\                        
Graphemes          & \multicolumn{1}{c|}{Lab} & \multicolumn{3}{c|}{Alv} & \multicolumn{1}{c|}{Vel} & \multicolumn{1}{c}{Glot} \\ \hline
N                  & m                        & n      &        &        &                          &                          \\
\multirow{2}{*}{P} & p                        & t      & ts     & tś     &                          &                          \\
                   & b                        & d      & dz     & dź     & g & k                        \\
F                  & f                        &        & s      & ś      &                          & h                        \\
L                  &                          &        & l      & y      & w                        &                         
\end{tabular}
\label{consonants}
\caption{Consonants of \lang{}}
\end{table}

The basic vowels of \lang{} are /i e a o u̥/, romanized simply as \orth{i e a o u}. Additionally, \lang{} contrasts vowel length on all five vowels. Additionally several diphthongs exist (and are to be distinguished from vowel sequences): the rising diphthongs /ai̯ aɨ̯ au̯/, spelled \orth{ai ae au}, and the falling diphthongs /ie̯ uo̯/, spelled \orth{ie uo}.

\subsection{Tone}
\lang{} has two tones: high and low. Underlyingly any syllable may be marked high or low, but there is an important rule determining the tone: a low tone between any two high tones within the same word is raised. Thus all high tones within a word form a plateau. Tone is marked with a macron over all high vowels. On dipthongs and long vowels, tone may change between the two segments, creating falls and rises on long syllables.

\subsection{Syllable Structure}
For the purpose of describing syllable structure, the following categories are relevant: 

\begin{itemize}
	\item N — Nasal consonants
	\item P — Plosives and affricates except for /ʔ/
	\item F — Fricatives
	\item L — Liquids /l j ɰ/
	\item V — Vowels (short, long and diphthongs)
\end{itemize}

Each syllable contains exactly one element from \textbf{V}, the nucleus. Optionally, it may have simple or complex onsets as well as simple codas. It follows a list of possible onsets with some examples: 

\begin{table}[H]
\centering
\begin{tabular}{llll}
∅ 	& \con{uk} 		&[u̥ʔ]			&\en{eye}    	\\
  	& \con{āēod}	&[áɨ̯́.ot]		&\en{fire} 		\\
N 	& \con{maek}	&[maɨ̯ʔ]			&\en{father}		\\    
	& \con{nuokeed} &[nuo̯ʔ.eːt]		&\en{ink brush}	\\
P	& \con{baumu} 	&[pau̯.m\f]		&\en{light overcoat}\\   
	& \con{tśēēmā}	&[tɕʰéːmá]		&\en{boat}		\\
PL	& \con{twainop} &[tʰɰai̯.nopʰ]	&\en{branch}		\\
	& \con{tśluoāk} &[tɕʰluo̯.áʔ]	&\en{coconut}	\\
F	& \con{sauthu}	&[sau̯tʰ.h\f]	&\en{flower}		\\
	& \con{seek}	&[seːʔ]			&\en{rain}		\\
FL	& \con{flufaek}	&[ɸl\f{}.ɸaɨ̯ʔ]	&\en{storm}		\\
	& \con{śwēēp}	&[ɕɰéːpʰ]		&\en{snake}
	
\end{tabular}
\end{table}

As for codas, the only allowed options are ∅, or any plosive or affricate, including /ʔ/. Examples of all possible coda kinds can be found in the list above as well.
