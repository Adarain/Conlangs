\documentclass[a4paper,abstracton]{scrartcl}
\usepackage{color}
\usepackage{tabu}
\usepackage{multirow}
\usepackage{multicol}

%\usepackage{fontspec}
\usepackage[hidelinks]{hyperref}
\usepackage{float}
\restylefloat{table}


\usepackage{graphicx}
\usepackage{expex}
\usepackage{libertine}

\newcommand{\gl}[1]{\textsc{#1}}
\newcommand{\en}[1]{``#1''}
%\newcommand{\vp}[1]{\hspace{0pt}{\color{teal}#1}}
\newcommand{\vp}[1]{\emph{#1}}
\newcommand{\orig}[1]{\emph{#1}}
\newcommand{\orth}[1]{⟨#1⟩}

%\renewcommand{\abstractname}{Abstract}


\title{Viossa}
\author{Sascha M. Baer}

\begin{document}
\maketitle
\begin{abstract}
The first section of this paper reproduces the contents of the authors presentation at the Eighth Language Creation Conference \emph{Ka Du Hanasu? \--- Collaborative Conpidgins}, in which the concept of the conpidgin is explained and examplified. Relations to pidgins and traditional conlangs are also examined. In the second section, the conpidgin Viossa is documented. 
\end{abstract}

\section{Ka du Hanasu? \--- Collaborative Conpidgins}
This section aims to reproduce the author’s presentation (of same title) at the Eighth Language Creation Conference, held at Anglia Ruskin University, Cambridge, on June 22nd, 2019. Beyond some minor details which had to be cut due to timing constraints, no information beyond what was said at the presentation is included in this section. The presentation was recorded and may be watched at %TODO link to presentation

\subsection{Introduction}
The following is an example sentence in Viossa, a language created by the author and several of his friends, most notably Klaus Gjika, Pauli J. Marttinen, Serena and Runar%TODO name stylization and permissions
\footnote{Many other people have participated in the project, listed here are the people who participated from the beginning and remained active throughout most of the project’s lifetime. Honorable mentions go to} %TODO honorable mentions 
starting in late 2014\footnote{Indeed, on Christmas Eve, after everyone had gotten bored of their families.}:

%TODO gloss
Imadag un vil hanu fu Viossa, glossa ka un auen mik dan mahha.

now-day I want speak of we-LANG, language REL 1s and friend PST make

“Today I want to speak about Viossa (lit. “we-ese”), the language that I and (my) friends have made.”
%

The above sentence may appear entirely unremarkable to the experienced conlanger. Not only is its syntax extremely European and its morphology nearly nonexistent, its vocabulary is almost entirely taken from various natural languages:
\begin{itemize}
\item \vp{imadag}, compounded from Japanese \orig{ima} \en{now} and Norwegian \orig{dag} \en{day}
\item \vp{un}, from Albanian \orig{unë} \en{I}
\item \vp{vil}, from Swiss German \orig{will} \en{want} (first/third person form)
\item \vp{hanu}, shortened from Japanese \orig{hanasu} \en{speak} 
\item \vp{fu}, from Swiss German \orig{vu} \en{of, from}
\item \vp{Viossa}, compounded from Norwegian \orig{vi} \en{we} and a suffix \vp{-ossa} derived from Greek \orig{glossa} \en{language}
\item \vp{glossa}, see immediately above
\item \vp{ka}, from Norwegian\footnote{Specifically, the Tromsø dialect} \orig{ka} \vp{what}
\item \vp{auen}, the only word in this sentence without an obvious etymology (elaborated later)
\item \vp{mik}, from Albanian \orig{mik} \en{friend}
\item \vp{dan}, from Norwegian \orig{da} \en{then}
\item \vp{mahha}, from Swiss German \orig{macha} \en{do, make} (infinitive or first person form)
\end{itemize}

This analysis, however, is misleading, as the language of this sentence is not a constructed language in the classical sense. Viossa is what the author has named a collaborative conpidgin. The next section explains this term by outlining the creation process of Viossa.



\end{document}